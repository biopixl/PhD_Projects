\documentclass[11pt,letterpaper]{article}
\usepackage[utf8]{inputenc}
\usepackage{textgreek}
\usepackage{graphicx}
\usepackage{natbib}
\usepackage{hyperref}
\usepackage{lineno}
\usepackage{setspace}
\onehalfspacing
\linenumbers

\title{Supplementary Material: Proposed Research Plan\\
\large Branch-Specific Phylogenomics to Identify Genetic Mechanisms of Dog Breed Formation}

\author{Isaac N. Aguilar Rivera}


\begin{document}

\maketitle

\begin{abstract}
This supplementary document details the proposed three-aim research program to functionally validate, expand, and integrate preliminary phylogenomic discoveries from the main manuscript. The proposed research includes: (1) functional validation of neurotransmitter and developmental signaling candidates through expression profiling, structural modeling, and cell-based assays, (2) expansion of the phylogenetic framework to include grey wolf when genome resources become available, and (3) integration with Dog10K population genomic data to identify breed-specific selection patterns. Total estimated budget: \$415,000--\$580,000 over 3 years.
\end{abstract}

\section{Proposed Research Plan}

Building on these preliminary findings, we propose a three-aim research program to validate, expand, and integrate our phylogenomic discoveries.

\subsection{Aim 1: Functional Validation of Neurotransmitter and Developmental Signaling Candidates}

\textbf{Rationale:} Preliminary data identified Tier 1 genes representing two distinct functional categories: neurotransmitter receptors (50\%, 3/6: \textit{GABRA3}, \textit{HTR2B}, \textit{HCRTR1}) and developmental signaling genes (3/6: \textit{FZD3}, \textit{FZD4}, \textit{EDNRB}). Experimental validation is necessary to establish causal links between dog-specific substitutions and phenotypic outcomes in both behavioral and morphological trait dimensions. High tractability scores suggest these candidates are amenable to immediate functional investigation across multiple systems.

\textbf{Approach:}

\textbf{1.1. Comparative expression profiling (RNA-seq).} We will quantify expression divergence at candidate loci across developmental stages and tissue types, with specific focus on brain region-specific analysis for neurotransmitter receptors. Tissues will include: \textbf{(a) Brain regions} for neurotransmitter candidates: amygdala (fear/aggression, \textit{GABRA3}), prefrontal cortex (social cognition, \textit{HTR2B}), hypothalamus (arousal/motivation, \textit{HCRTR1}); \textbf{(b) Embryonic samples} (E14-E18, critical for neural crest development, \textit{FZD3}, \textit{FZD4}, \textit{EDNRB}); \textbf{(c) Adult tissues}: skin (pigmentation), craniofacial structures (morphological diversification). Comparisons will contrast 5 phenotypically diverse dog breeds (e.g., Chihuahua, Great Dane, Border Collie, Bulldog, Greyhound) against dingo and wolf (when available). RNA-seq will be performed at 30M reads/sample depth using Illumina NovaSeq. Differential expression analysis will use DESeq2, identifying genes with significant expression divergence (FDR $<$ 0.05, $\mid$log$_2$FC$\mid$ $>$ 1) at candidate loci.

\textbf{1.2. Protein structure modeling.} Dog-specific substitutions at Tier 1 genes will be mapped to 3D protein structures using AlphaFold2. Predicted structures will be compared to ancestral (dingo/wolf) sequences to identify conformational changes. Molecular dynamics simulations (100 ns trajectories, GROMACS) will assess stability and functional impacts. Substitutions affecting protein-protein interaction interfaces, catalytic sites, or regulatory domains will be prioritized for experimental validation. We anticipate identifying 10-15 high-priority variants with predicted functional consequences.

\textbf{1.3. Cell-based functional assays (pilot).} To establish proof-of-concept for functional effects across both neurotransmitter and developmental pathways, we will conduct multiple assay systems:

\textbf{(a) Neurotransmitter receptor activity assays:}
\begin{itemize}
\item \textbf{GABA receptor function}: Electrophysiology (patch-clamp) or GABA-induced chloride flux measurements comparing dog vs. ancestral \textit{GABRA3} alleles expressed in HEK293 cells or Xenopus oocytes
\item \textbf{Serotonin receptor signaling}: G-protein coupled receptor (GPCR) activation assays (cAMP, IP$_3$ measurements) comparing dog vs. ancestral \textit{HTR2B} variants in mammalian cell lines
\item \textbf{Orexin receptor function}: Calcium mobilization or cAMP assays for \textit{HCRTR1} comparing dog vs. ancestral alleles
\end{itemize}

\textbf{(b) Wnt pathway activity} (retain original plan):
\begin{itemize}
\item TOPFlash/FOPFlash reporter assays comparing dog vs. ancestral alleles of \textit{FZD3} and \textit{FZD4}
\item Transfection into HEK293 cells, quantification via dual-luciferase assay
\end{itemize}

Significant activity differences ($>$20\% change, \textit{p} $<$ 0.01) will support functional impacts of dog-specific substitutions.

\textbf{Expected Outcomes:}
\begin{itemize}
\item Expression signatures for \textbf{both neurotransmitter and developmental genes} across brain regions and developmental stages
\item Structural predictions for 10-15 high-priority substitutions across neurotransmitter receptors and developmental signaling genes
\item Pilot functional data demonstrating altered receptor activity for neurotransmitter genes (GABA, serotonin, orexin) and Wnt pathway modulation for developmental genes
\item Comparative evidence for \textbf{parallel selection on behavioral AND morphological pathways}
\item Manuscript: "Multi-pathway validation reveals neurotransmitter and developmental targets of dog breed formation"
\end{itemize}

\textbf{Timeline:} 18-24 months

\textbf{Budget:} \$150,000-\$200,000 (RNA-seq library prep and sequencing, computational resources for AlphaFold2/MD, cell culture reagents, reporter assays)

\subsection{Aim 2: Expanded Phylogenetic Framework with Wolf Genome}

\textbf{Rationale:} The Greenland wolf genome (\textit{Canis lupus orion}) is undergoing annotation via the Ensembl pipeline at EBI, with anticipated public release in 2025-2026. Addition of the wolf genome will enable a four-species framework: (((\textit{Dog}, \textit{Dingo}), \textit{Wolf}), \textit{Fox}), providing temporal layering of selection signals.

\textbf{Approach:}

\textbf{2.1. Re-analysis with 4-species phylogeny.} Upon wolf genome availability, we will apply the same aBSREL pipeline to identify: (a) breed formation signals (dog branch only, expected to refine current 401 genes), (b) early domestication signals (dog+dingo clade vs. wolf, expected 50-100 genes), and (c) wild canid evolution (wolf branch, serving as control). Expected dataset: $\sim$17,000 genes with 4-way orthologs based on current Ensembl Compara coverage.

\textbf{2.2. Temporal partitioning of selection signals.} We will classify genes by temporal layer: (1) breed-specific (dog only), (2) domestication-associated (dog+dingo shared), (3) lineage-specific background (wolf). Functional enrichment analysis will be performed separately for each category to identify pathway-level themes specific to evolutionary stages.

\textbf{2.3. Validation of 3-species findings.} Current 401 candidates will be cross-validated with 4-species results. We expect a subset (estimated 250-300 genes) to retain strong dog-specific signal, while others may reclassify as shared domestication signatures. This refinement will increase confidence in breed-specific candidates.

\textbf{2.4. Power analysis and simulation.} We will perform statistical power analysis to assess detection sensitivity for the wolf branch given estimated divergence time ($\sim$15-40 KYA) and expected sequence divergence. Simulations will quantify false positive/negative rates under various selection scenarios ($\omega$ = 1.5-5, 5-20\% sites under selection).

\textbf{Expected Outcomes:}
\begin{itemize}
\item Refined set of 250-300 high-confidence breed-specific candidates
\item Identification of 50-100 early domestication genes (dog+dingo branch)
\item Partitioned functional themes by evolutionary layer (Wnt enrichment specific to breed formation vs. shared with domestication)
\item Power analysis framework applicable to other domestication systems
\item Manuscript: "Temporal layering of selection during dog domestication: insights from four-species phylogenomics"
\end{itemize}

\textbf{Timeline:} 12-18 months (contingent on wolf genome release, estimated mid-2025 to early-2026)

\textbf{Budget:} \$75,000-\$100,000 (computational resources for 17,000-gene aBSREL analysis, data storage, validation bioinformatics, personnel)

\subsection{Aim 3: Integration with Dog10K Population Genomic Data}

\textbf{Rationale:} The Dog10K Consortium has generated whole-genome sequences for 2,000+ canids including 1,611 dogs across 321 breeds \citep{Dog10K2023, Dog10K2024}. Integration of our phylogenetic candidates (401 genes) with population-level data will reveal breed-specific vs. pan-breed selection patterns and enable genotype-phenotype associations.

\textbf{Approach:}

\textbf{3.1. Retrieve Dog10K haplotype data for candidate genes.} We will download VCF files for the 401 gene regions from the Dog10K database (https://dog10k.kiz.ac.cn/). Haplotype phasing will be performed using SHAPEIT4, and allele frequencies will be calculated for each breed.

\textbf{3.2. Haplotype-based selection scans.} Using Dog10K integrated tools, we will perform selection scans using multiple methods: integrated haplotype score (iHS), cross-population extended haplotype homozygosity (XP-EHH), and number of segregating sites by length (nSL). These methods detect recent selective sweeps via extended haplotype homozygosity. Candidates with significant signals (top 1\% genome-wide) in multiple breeds will be classified as pan-breed targets, while breed-specific signals will identify lineage-restricted selection.

\textbf{3.3. Genotype-phenotype association.} We will cross-reference candidate genes with AKC breed standard phenotypes (size, coat color/texture, craniofacial morphology, behavioral traits). For Wnt pathway genes, we will focus on craniofacial and pigmentation traits given known pathway roles. Association tests will use breed-level trait scores regressed against haplotype frequencies, controlling for population structure via principal components.

\textbf{3.4. Concordance analysis.} We will assess concordance between phylogenetic aBSREL signals and population-level sweep signals. Genes with strong support from both methods (estimated 50-100 genes) will represent high-confidence candidates with convergent evolutionary evidence. Discordant cases will be investigated to understand different evolutionary modes (e.g., soft vs. hard sweeps, standing variation vs. new mutations).

\textbf{Expected Outcomes:}
\begin{itemize}
\item Breed-specific sweep map for 401 phylogenetic candidates across 321 breeds
\item Subset of 50-100 genes with convergent phylogenetic + population evidence
\item Genotype-phenotype associations for Wnt pathway genes (focus on craniofacial/pigmentation)
\item Integrated candidate ranking system combining phylogenetic, population, and phenotypic data
\item Manuscript: "Integration of phylogenomic and population genomic approaches reveals convergent selection during dog breed formation"
\end{itemize}

\textbf{Timeline:} 12 months

\textbf{Budget:} \$50,000-\$75,000 (Dog10K data access and storage, computational analysis for 321 breeds, population genetics software, phenotype database compilation)

\subsection{Timeline and Milestones}

\textbf{Year 1 (Months 1-12):}
\begin{itemize}
\item Aim 1: RNA-seq sample collection, library prep, sequencing (Months 1-6)
\item Aim 1: Expression analysis and initial structural modeling (Months 6-9)
\item Aim 3: Dog10K data retrieval and haplotype-based scans (Months 3-9)
\item Aim 3: Genotype-phenotype associations (Months 9-12)
\item Milestone: First manuscript submission (Aim 3, Month 12)
\end{itemize}

\textbf{Year 2 (Months 13-24):}
\begin{itemize}
\item Aim 1: Complete structural predictions and MD simulations (Months 13-18)
\item Aim 1: Wnt reporter assays (pilot functional validation) (Months 18-21)
\item Aim 2: Wolf genome monitoring, pipeline preparation (Months 13-24)
\item Aim 2: 4-species aBSREL analysis upon wolf release (Months 18-24, contingent)
\item Milestone: Second manuscript submission (Aim 1, Month 21)
\end{itemize}

\textbf{Year 3 (Months 25-36):}
\begin{itemize}
\item Aim 2: Complete 4-species analysis and temporal partitioning (Months 25-30)
\item Integration: Synthesis across all three aims (Months 30-33)
\item Manuscript preparation: Aim 2 + integrated synthesis papers (Months 30-36)
\item Milestone: Final manuscripts submitted (Months 33-36)
\end{itemize}

\subsection{Budget Summary}

\begin{table}[h]
\centering
\begin{tabular}{lrr}
\hline
\textbf{Category} & \textbf{Amount} & \textbf{Notes} \\
\hline
Personnel & \$180,000-\$240,000 & Graduate student or postdoc (1.0 FTE, 3 years) \\
          & \$15,000-\$20,000 & Undergraduate research assistants (summers) \\
Computational & \$10,000-\$15,000 & HPC cluster time (4-way aBSREL analysis) \\
              & \$5,000-\$10,000 & Data storage (Dog10K integration) \\
Experimental & \$75,000-\$100,000 & RNA-seq (tissue collection, library prep, sequencing) \\
             & \$10,000-\$15,000 & Protein structural modeling (AlphaFold2, MD) \\
             & \$25,000-\$35,000 & Cell-based assays (Wnt reporters, reagents) \\
Travel & \$10,000-\$15,000 & Conference presentations (3 years) \\
Publications & \$5,000-\$10,000 & Open-access fees (4-5 manuscripts) \\
\hline
\textbf{Subtotal} & \$335,000-\$460,000 & \\
Indirect costs & \$80,000-\$120,000 & University overhead (varies by institution) \\
\hline
\textbf{Total} & \$415,000-\$580,000 & 3-year program \\
\hline
\end{tabular}
\caption{Estimated budget for proposed research program}
\end{table}


\bibliographystyle{plainnat}
\bibliography{references}

\end{document}
