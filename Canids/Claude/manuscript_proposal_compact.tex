\documentclass[11pt,letterpaper]{article}
\usepackage[utf8]{inputenc}
\usepackage{textgreek}
\usepackage[utf8]{inputenc}
\usepackage{graphicx}
\usepackage{natbib}
\usepackage{hyperref}
\usepackage{lineno}
\usepackage{setspace}
\onehalfspacing
\linenumbers

\title{Branch-Specific Phylogenomics to Identify Genetic Mechanisms of Dog Domestication}

\author{Isaac N. Aguilar Rivera}


\begin{document}

\maketitle

\begin{abstract}
Dog domestication represents one of the earliest and most extensive cases of human-mediated phenotypic diversification, yet a critical question remains unresolved: which genomic changes underlie early domestication (transition from wolf to dog) versus breed formation (diversification of modern breeds)? We propose to address this gap using a comparative phylogenomic framework that distinguishes temporal layers of selection. Our three-species analysis using the adaptive Branch-Site Random Effects Likelihood (aBSREL) model identified genes under episodic positive selection uniquely in domestic dogs (\textit{C. lupus familiaris}) relative to dingoes (\textit{C. lupus dingo}), with red fox (\textit{Vulpes vulpes}) as outgroup. Across 17,046 orthologous coding sequences, 430 genes (2.52\%) showed significant dog-specific positive selection. Functional enrichment identified protein binding as the most significant category ($p=0.0037$, 117 genes). Specific pathways included Wnt signaling ($p=0.041$, 16 genes) and multiple neurotransmitter systems. Multi-criteria prioritization identified six high-confidence candidates representing three functional categories (neurotransmitter receptors, Wnt receptors, and neural crest development), suggesting parallel selection on behavioral regulation pathways alongside morphological diversification. We propose to: (1) functionally validate neurotransmitter and developmental signaling candidates through expression profiling and structural modeling, (2) expand the phylogenetic framework to include the grey wolf (\textit{Canis lupus lupus}), and (3) integrate findings with Dog10K population genomic data to identify breed-specific vs. pan-breed selection patterns. This work will provide genome-wide insights into evolutionary forces shaping dog diversification and establish a mechanistic foundation for understanding rapid phenotypic evolution in domesticated species.
\end{abstract}


\section{Introduction}

Dog (\textit{Canis lupus familiaris}) domestication, dating to 15,000-40,000 years ago, resulted in remarkable phenotypic diversification across morphology, behavior, physiology, and cognition—collectively termed the "domestication syndrome" \citep{Darwin1868, Belyaev1979, Wilkins2014, Freedman2014, Frantz2016}. However, a critical question remains unresolved: which genomic changes underlie early domestication versus breed formation? Traditional comparisons between dogs and wolves conflate 15,000-40,000 years of evolutionary change, limiting mechanistic inference. Australian dingoes (\textit{Canis lupus dingo}) provide a unique evolutionary reference, retaining key domestication-associated phenotypes while lacking modern breed specializations. Having diverged after initial domestication but prior to intensive breed formation (8,000–10,000 years ago), dingoes enable detection of genomic signals specific to breed diversification rather than early domestication \citep{Cairns2022, Savolainen2004}. This complements Dog10K Consortium population-level efforts \citep{Dog10K2023, Dog10K2024}, which lack phylogenetic temporal resolution.

We employed a three-species phylogeny (dog, dingo, red fox) with the adaptive Branch-Site Random Effects Likelihood (aBSREL) model \citep{Smith2015} to isolate lineage-specific selection on the dog branch, explicitly distinguishing early domestication signals from breed formation pressures. Our preliminary analysis of 17,046 genes identified 430 candidates under dog-specific positive selection, revealing multiple biological themes including protein binding, neurotransmitter signaling, and developmental pathways. Following recent calls for broader discovery-oriented approaches in evolutionary genomics \citep{Ostrander2024}, we combine genome-wide selection analyses with Multi-Criteria Decision Analysis (MCDA) to translate statistical evidence into biologically meaningful insights.

The research question our project addresses: what genomic changes distinguish breed-associated artificial selection from early domestication processes in dogs, and can these signals reveal the molecular basis of rapid phenotypic diversification?

We propose a three-aim research program to: \textbf{(1)} functionally validate neurotransmitter and developmental signaling candidates through expression profiling and structural modeling, \textbf{(2)} expand the phylogenetic framework to include gray wolf when genome resources become available, and \textbf{(3)} integrate findings with Dog10K population genomic data to identify breed-specific vs. pan-breed selection patterns. This work will provide genome-wide insights into evolutionary forces shaping dog diversification and establish a mechanistic foundation for understanding rapid phenotypic evolution across domesticated species.


\section{Methods}

\subsection{Genome Data and Ortholog Identification}

We obtained reference genomes from Ensembl release 115: \textit{C. l. familiaris} (CanFam4.0/ROS\_Cfam\_1.0), \textit{C. l. dingo} (ASM325472v1), and \textit{V. vulpes} (VulVul2.2). Orthologous genes were identified using Ensembl Compara with high-confidence one-to-one orthologs. Coding sequences (CDS) were retrieved via the Ensembl REST API, yielding 17,078 genes present across all three species.

\subsection{Sequence Alignment and Phylogeny}

Protein sequences were aligned using MAFFT v7.505 with the L-INS-i algorithm. Alignments were back-translated to codon sequences with PAL2NAL v14. Genes with $<30\%$ aligned sequence coverage, $<50\%$ pairwise identity, or internal stop codons were excluded, resulting in 17,046 high-quality codon alignments. All analyses used the fixed species topology ((\textit{C. familiaris}, \textit{C. dingo}), \textit{V. vulpes}).

\subsection{Detection of Lineage-Specific Positive Selection}
Episodic positive selection on the dog lineage was tested using the adaptive Branch-Site Random Effects Likelihood (aBSREL) model implemented in HyPhy v2.5.59. aBSREL estimates branch-specific distributions of nonsynonymous-to-synonymous substitution rate ratios ($\omega$), permitting $\omega > 1$ for a subset of sites. Likelihood ratio tests compared models allowing versus prohibiting positive selection on the dog branch. Bonferroni correction for 17{,}046 tests set a significance threshold of $\alpha = 2.93 \times 10^{-6}$. Genes were classified as dog-specific if significant support for $\omega>1$ occurred exclusively on the dog branch.

\subsection{Quality Control and Validation}
Analytical validity was assessed by computing the genomic inflation factor ($\lambda$), defined as the ratio of observed to expected median $\chi^2$ statistics \citep{Devlin1999}. Annotation completeness was quantified using Ensembl BioMart. Chromosomal distributions of significant genes were evaluated using a $\chi^2$ goodness-of-fit test.

\subsection{Functional Annotation and Enrichment Analysis}
Functional annotation was performed using the Ensembl BioMart API. Enrichment analysis was conducted with g:Profiler (g:GOSt) using Fisher's exact test and g:SCS multiple-testing correction, with all 17{,}046 orthologs as the background set \citep{Raudvere2019}. Gene Ontology terms, pathways, and phenotype categories were considered significant if their adjusted p-values after correction was $<0.05$.

\subsection{Hybrid Gene Prioritization Framework}
To identify high-value candidates for follow-up investigation, we applied a hybrid prioritization strategy integrating hypothesis-driven and data-driven approaches. The hypothesis-driven component used Multi-Criteria Decision Analysis (MCDA) scoring with four weighted components: (1) selection strength (0--10 points; scaled $-\log_{10} P$), (2) biological relevance (0--3 points; involvement in domestication-related pathways), (3) experimental tractability (0--3 points; assay and model feasibility), and (4) literature support (0--4 points; prior associations and citation density). Genes with total MCDA scores $\geq$ 16 were assigned Tier 1 (high priority). To complement pathway-based prioritization and enable unbiased discovery, genes meeting extreme significance criteria (\textit{p} $<$ 10$^{-12}$ AND $\omega > 1$) were automatically elevated to Tier 1 regardless of MCDA score.

\section{Preliminary Results}

\subsection{Quality Control and Model Performance}

Across the 17{,}046 orthologous coding sequences included in the final dataset, aBSREL identified 430 genes (2.52\%) with evidence of episodic positive selection on the dog branch after Bonferroni correction (\textit{p} $<$ 2.93$\times$10$^{-6}$).


\begin{figure}[htbp]
\centering
\includegraphics[width=\textwidth]{figures/Figure1_QualityControl.png}
\caption{\textbf{Quality control and analytical performance metrics.} We applied the adaptive Branch-Site Random Effects Likelihood (aBSREL) model implemented in HyPhy v2.5.59 to 17,046 one-to-one orthologous genes. aBSREL uses likelihood ratio tests to compare models allowing versus prohibiting positive selection ($\omega > 1$) on specified branches. \textbf{(A)} Quantile-quantile plot of observed vs. expected $-\log_{10}$(p-values) under the null hypothesis of no selection. The genomic inflation factor ($\lambda = 67.7$) reflects extensive deviation from neutral expectations. \textbf{(B)} Annotation coverage among 430 candidate genes passing Bonferroni correction ($\alpha = 2.93 \times 10^{-6}$), showing 78\% with functional annotation from Ensembl BioMart. \textbf{(C)} Distribution of selection significance (p-values) stratified by annotation status. \textbf{(D)} Distribution of gene-wide $\omega$ estimates (dN/dS ratio) across all candidate genes (median = 0.68).}
\label{fig:qc}
\end{figure}

The elevated genomic inflation factor ($\lambda = 67.7$) is consistent with extensive deviation from neutral expectations across multiple loci, as expected given strong artificial selection during breed formation (Figure~\ref{fig:qc}A). Annotation coverage approached 80\% (Figure~\ref{fig:qc}B), and no significant differences in selection significance were observed between annotated and unannotated genes (Figure~\ref{fig:qc}C). The distribution of gene-wide $\omega$ values (median 0.68, Figure~\ref{fig:qc}D) reflects the expected pattern wherein aBSREL detects site-specific positive selection while gene-wide averages remain below 1, characteristic of episodic selection within constrained loci.


\subsection{Chromosomal Distribution of Candidate Genes}

Candidate genes were distributed across all dog chromosomes without strong evidence of clustering (Figure~\ref{fig:chromosome}A). A chi-square test comparing observed to expected counts yielded $\chi^{2} = 58.3$ (\textit{p} = 0.0186), a modest deviation likely attributable to chromosome size variation rather than concentration on specific chromosomal regions.

\begin{figure}[htbp]
\centering
\includegraphics[width=\textwidth]{figures/Figure2_ChromosomeDistribution.png}
\caption{\textbf{Chromosomal distribution of genes inferred to be under positive selection.} Candidate genes were mapped to dog chromosomes (CanFam4.0) using Ensembl coordinates. Chromosomal clustering was assessed using chi-square goodness-of-fit test comparing observed to expected counts based on chromosome size (number of genes). Proportions were normalized by dividing selected genes per chromosome by total genes per chromosome. \textbf{(A)} Raw counts of genes under selection across 38 autosomes and X chromosome (range: 6-47 genes). \textbf{(B)} Proportion of selected genes normalized by chromosome size, revealing approximately uniform $\sim$1.5-2\% selection rate across all chromosomes ($\chi^{2} = 58.3$, \textit{p} = 0.0186). \textbf{(C)} Genomic positions of 430 candidate genes plotted along chromosomes (karyotype-style), with red points indicating significant selection (\textit{p} $<$ 2.93$\times$10$^{-6}$) and gray points showing non-significant genes.}
\label{fig:chromosome}
\end{figure}

Normalization by chromosome size resulted in an approximately uniform proportion of selected genes ($\sim$1.5\%) across the autosomes and X chromosome (Figure~\ref{fig:chromosome}B), consistent with a broadly polygenic signature. The dispersed genomic positions (Figure~\ref{fig:chromosome}C) further support selection on distributed loci rather than chromosomal hotspots.


\subsection{Magnitude and Distribution of Selection Signals}

Among the 430 candidate genes, 254 (59.1\%) showed extremely small \textit{p}-values (\textit{p} $<$ 1$\times$10$^{-10}$; Figure~\ref{fig:selection}A). Despite these strong significance levels, gene-wide $\omega$ estimates remained below 1 for most genes (median = 0.68), consistent with episodic selection acting on localized sites rather than pervasive positive selection across entire coding regions (Figure~\ref{fig:selection}B).

\begin{figure}[htbp]
\centering
\includegraphics[width=\textwidth]{figures/Figure3_SelectionResults.png}
\caption{\textbf{Summary of branch-specific selection statistics.} For each gene, aBSREL estimates a gene-wide $\omega$ value (average across sites and branches) and reports a likelihood ratio test p-value for episodic positive selection on the dog branch. Genes were classified by significance thresholds (\textit{p} $<$ 10$^{-6}$, \textit{p} $<$ 10$^{-10}$, etc.) to assess strength of selection signals. \textbf{(A)} Volcano plot showing relationship between gene-wide $\omega$ (x-axis) and selection significance (y-axis, $-\log_{10}$ p-value). \textbf{(B)} Density distribution of gene-wide $\omega$ values across all 17,046 genes, with candidate genes (red) showing median $\omega$ = 0.68. \textbf{(C)} Frequency distribution of genes across significance thresholds, showing 254/430 candidates (59.1\%) with extremely significant \textit{p} $<$ 10$^{-10}$.}
\label{fig:selection}
\end{figure}

The pattern of low gene-wide $\omega$ with highly significant selection tests aligns with expectations for developmental and regulatory genes, which are typically constrained at most sites but may undergo episodic adaptive changes at specific residues. The high frequency of extremely significant p-values (Figure~\ref{fig:selection}C) demonstrates robust statistical support for the identified candidates.

\subsection{Functional Annotation and Pathway Enrichment}

Functional enrichment analysis using all 17{,}046 orthologs as background identified multiple significant Gene Ontology terms spanning molecular function, biological process, and cellular component categories (Figure~\ref{fig:wnt}A). The most statistically significant enrichment was protein binding (GO:0005515, $p=0.0037$, FDR=0.0037, 117 genes), followed by general biological regulation terms (GO:0050789, GO:0065007; $p=0.0068$; 179--184 genes) and cytoplasm (GO:0005737; $p=0.0098$; 172 genes). Additional enrichments included endoplasmic reticulum membrane localization (GO:0042175, GO:0005789, GO:0098827; $p=0.019$--0.023; 23--24 genes), mitochondrial genome maintenance (GO:0000002; $p=0.032$; 5 genes), positive regulation of cellular process (GO:0048522; $p=0.032$; 92 genes), and Wnt signaling pathway (GO:0016055, $p=0.041$, FDR=0.041; 16 genes).

\begin{figure}[htbp]
\centering
\includegraphics[width=\textwidth]{figures/Figure4_WntEnrichment.png}
\caption{\textbf{Functional enrichment of genes under positive selection.} Functional enrichment was performed using g:Profiler (g:GOSt) with Fisher's exact test and g:SCS multiple-testing correction. Background set: all 17,046 orthologous genes. Significance threshold: FDR $<$ 0.05. \textbf{(A)} Top enriched Gene Ontology terms among 430 candidate genes, sorted by statistical significance. Gene counts and $-\log_{10}$(p-values) shown for each term. \textbf{(B)} Selection strength (gene-wide $\omega$) for nine Tier 1 candidates, grouped by functional category. Dashed line indicates neutral selection ($\omega=1$). \textbf{(C)} Functional theme comparison across all 430 candidates, plotting median $\omega$ vs. median $-\log_{10}$(p-value) for six functional categories. Bubble size represents gene count; dashed lines indicate significance (p $<$ 10$^{-6}$) and neutral selection thresholds.}
\label{fig:wnt}
\end{figure}

The diversity of enriched categories suggests that breed formation involved selection on multiple biological processes rather than convergence on a single pathway. Protein binding hub genes (117 genes, \textit{p}=0.0037) may coordinate pleiotropic effects across pathways, while specific signaling systems contribute to distinct domestication syndrome traits. We focus detailed analysis on four emergent themes: (1) neurotransmitter signaling, (2) neural crest development, (3) protein interaction networks, and (4) developmental signaling. Neurotransmitter genes showed moderate gene-wide $\omega$ values (median = 0.68) with strong statistical support (Figure~\ref{fig:wnt}B,C), while Wnt pathway genes (16 genes, median $\omega$ = 0.68) exhibited similar patterns consistent with episodic selection on localized sites.

\subsection{Prioritization of Candidate Genes}

Application of the hybrid prioritization framework yielded nine Tier~1 genes (Figure~\ref{fig:prioritization}A): six via hypothesis-driven MCDA scoring ($\geq$16 points) and three via data-driven auto-elevation (\textit{p} $<$ 10$^{-12}$ AND $\omega > 1$).

\begin{figure}[htbp]
\centering
\includegraphics[width=\textwidth]{figures/Figure5_GenePrioritization.png}
\caption{\textbf{Prioritization outcomes using hybrid framework.} Hybrid prioritization combined hypothesis-driven MCDA (scores $\geq$16) with data-driven auto-elevation (\textit{p} $<$ 10$^{-12}$ AND $\omega > 1$). \textbf{(A)} Distribution of 430 candidate genes across priority tiers, showing 9 Tier 1 genes, 44 Tier 2 genes, and 377 Tier 3 genes. \textbf{(B)} Component scores averaged across genes within each tier. \textbf{(C)} Characteristics of nine Tier 1 genes showing total scores, functional annotations, and prioritization tracks. \textbf{(D)} Scatter plot of total prioritization score vs. selection strength.}
\label{fig:prioritization}
\end{figure}

Tier~1 genes included four neurotransmitter signaling genes (\textit{GABRA3} [GABA-A receptor α3], \textit{HTR2B} [serotonin receptor 2B], \textit{HCRTR1} [orexin receptor 1], \textit{SLC6A4} [serotonin transporter]), two neural crest genes (\textit{TFAP2B} [neural crest transcription factor AP-2β], \textit{EDNRB} [endothelin receptor type B]), two Wnt receptors (\textit{FZD3}, \textit{FZD4}), and one craniofacial signaling gene (\textit{FGFR2} [FGF receptor 2]). Notably, \textit{TFAP2B} ($\omega$=1.20) and \textit{SLC6A4} ($\omega$=1.12) were auto-elevated despite lower MCDA scores, capturing strong selection on genes outside predefined pathways. \textit{TFAP2B} regulates neural crest specification and craniofacial development, while \textit{SLC6A4} controls serotonin reuptake and has established associations with canine behavioral phenotypes. \textit{FGFR2}, though below the auto-elevation threshold, qualified as an alternative hypothesis candidate due to established roles in skull morphology.

Neurotransmitter signaling genes constitute 44\% of Tier 1 candidates (4/9), representing three independent systems (GABAergic, serotonergic, orexinergic). These provide molecular substrates for behavioral domestication syndrome traits including reduced aggression (\textit{GABRA3}), altered social behavior (\textit{HTR2B}, \textit{SLC6A4}), and human-directed attention (\textit{HCRTR1}). Neural crest genes (\textit{TFAP2B}, \textit{EDNRB}) link behavioral and morphological changes through shared developmental origins.

The hybrid framework successfully captured both pathway-predicted candidates (MCDA track) and extreme signals outside predefined categories (auto-elevation track), revealing parallel selection on behavioral (neurotransmitter) and morphological (developmental) trait dimensions. This balanced discovery approach argues against single-pathway explanations and demonstrates that rapid phenotypic evolution during breed formation involved functionally diverse mechanisms.


\section{Discussion}

A central finding is the identification of multiple biological themes among genes under selection, rather than convergence on a single pathway. Neurotransmitter signaling genes (\textit{GABRA3}, \textit{HTR2B}, \textit{HCRTR1}, \textit{SLC6A4}) constitute 44\% of Tier 1 candidates (4/9), representing three independent systems (GABAergic, serotonergic, orexinergic) that provide molecular substrates for behavioral domestication syndrome traits. \textit{GABRA3} mediates inhibitory neurotransmission and anxiety regulation; selection on this gene may underlie reduced fear responses. GABAergic signaling alterations have been documented in domesticated foxes selected for tameness \citep{Trut2009}. \textit{HTR2B} and \textit{SLC6A4} regulate serotonergic signaling controlling mood and social behavior; altered serotonin pathways distinguish domesticated versus wild foxes \citep{Kukekova2018}. \textit{HCRTR1} controls wakefulness and reward processing, potentially relevant to human-directed attention; notably, \textit{HCRTR1} mutations cause narcolepsy in dogs \citep{Lin1999}. \textit{TFAP2B} ($\omega$=1.20, strongest signal) regulates neural crest specification, linking behavioral and morphological changes through shared developmental origins. The convergence of neurotransmitter/neural signaling genes—including the highest $\omega$ values genome-wide—suggests that behavioral domestication syndrome has a prominent genetic basis in receptor-level and neural developmental changes.

Protein binding, the most statistically significant enrichment (117 genes, $p=0.0037$), may reflect evolution of hub proteins coordinating pleiotropic trait changes across multiple pathways. Wnt signaling showed significant enrichment (GO:0016055, $p=0.041$, 16 genes), consistent with neural crest hypotheses \citep{Wilkins2014}, though it ranked 11th among significant GO terms, suggesting it represents one component of broader polygenic architecture. Our identification of neural crest genes (\textit{EDNRB} [migration/pigmentation], \textit{TFAP2B} [specification]), Wnt receptors (\textit{FZD3}, \textit{FZD4}), and \textit{FGFR2} (craniofacial morphology) among top candidates provides support for developmental pathway contributions to morphological traits. However, the 44\% representation of neurotransmitter signaling genes, including the highest $\omega$ value (\textit{SLC6A4}), demonstrates that behavioral domestication required complementary mechanisms beyond developmental pathway rewiring. This multi-pathway architecture encompassing neurotransmitter signaling, neural development, protein interaction networks, and morphological pathways argues against monolithic explanations and suggests that rapid phenotypic evolution during breed formation involved parallel selection on functionally diverse loci rather than coordinated changes in a single developmental program.


\section{Conclusions and Future Directions}

Our preliminary three-species phylogenomic framework has identified 430 genes with evidence of episodic positive selection specific to the dog lineage, revealing multiple biological themes rather than a single dominant pathway. Functional enrichment identified protein binding as the most statistically significant category (117 genes, $p=0.0037$), followed by general biological regulation, endoplasmic reticulum localization, and specific pathways including neurotransmitter signaling and Wnt signaling ($p=0.041$, 16 genes). Hybrid prioritization combining hypothesis-driven MCDA with data-driven auto-elevation identified nine Tier 1 candidates, with neurotransmitter signaling genes constituting 44\% (4/9): \textit{GABRA3}, \textit{HTR2B}, \textit{HCRTR1}, and \textit{SLC6A4} ($\omega$=1.12). Neural crest genes (\textit{TFAP2B} [$\omega$=1.20, strongest signal genome-wide], \textit{EDNRB}) and developmental signaling genes (\textit{FZD3}, \textit{FZD4}, \textit{FGFR2}) suggest parallel selection on behavioral and morphological trait dimensions.

Most candidate genes exhibited low gene-wide $\omega$ but strong statistical support for localized adaptive change, consistent with selection acting on constrained developmental and regulatory loci. The dispersed chromosomal distribution of selected genes and the elevated genomic inflation factor together suggest a broadly polygenic pattern consistent with the diversity of phenotypes targeted during dog breed formation. The convergence of neurotransmitter/neural signaling genes—including the two highest $\omega$ values genome-wide (\textit{TFAP2B}, \textit{SLC6A4})—alongside developmental pathway genes demonstrates that breed formation involved functionally diverse mechanisms for behavioral versus morphological domestication syndrome components. The hybrid framework successfully captured both pathway-predicted candidates and extreme signals outside predefined categories, enabling balanced discovery.

We propose a three-aim research program that will: (1) establish functional causality for both neurotransmitter/neural and developmental signaling candidates through brain region-specific expression profiling, structural modeling, and receptor activity assays, (2) achieve unprecedented temporal resolution via four-species phylogenomics when the wolf genome becomes available, and (3) integrate phylogenetic discoveries with Dog10K population data for convergent validation across behavioral and morphological traits. This work will transform preliminary genomic signals into mechanistic understanding of the evolutionary forces shaping rapid phenotypic diversification across multiple trait dimensions.

The anticipated availability of the gray wolf genome in 2026 creates a time-sensitive opportunity to distinguish early domestication signals from breed formation signals. This temporal layering, combined with functional validation of multi-pathway candidates and integration with population genomic data, will provide a comprehensive view of the genetic architecture underlying one of the most extensive cases of human-mediated phenotypic evolution. The multi-pathway framework established through this work encompassing neurotransmitter signaling, protein interaction networks, and developmental pathways represents a more complete model of domestication genetics than single-pathway explanations.

These results will inform broader questions in evolutionary biology regarding the genetic basis of rapid adaptation, the role of pleiotropic pathways in correlated trait evolution, and the mechanisms enabling parallel selection on behavioral and morphological dimensions. The framework established here will be applicable to understanding domestication processes across species and will contribute to translational applications in veterinary medicine, behavioral genetics, and conservation biology.

\bibliographystyle{plainnat}
\bibliography{references}

\end{document}
