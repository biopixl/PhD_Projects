\documentclass[11pt,letterpaper]{article}
\usepackage[utf8]{inputenc}
\usepackage{graphicx}
\usepackage{natbib}
\usepackage{hyperref}
\usepackage{lineno}
\usepackage{setspace}
\onehalfspacing
\linenumbers

\title{Genome-wide Positive Selection Analysis Identifies Wnt Signaling and Neurotransmitter Receptor Evolution During Dog Breed Formation}

\author{Isaac N. Aguilar Rivera}

\date{\today}

\begin{document}

\maketitle

\begin{abstract}
Dog domestication represents one of the earliest examples of human-mediated evolution, yet the genomic basis of breed-specific traits remains incompletely characterized. We performed a three-species phylogenetic comparative analysis using adaptive Branch-Site Random Effects Likelihood (aBSREL) to identify genes under positive selection exclusively in domestic dogs (\textit{Canis lupus familiaris}) but not dingoes (\textit{Canis lupus dingo}), using red fox (\textit{Vulpes vulpes}) as outgroup. This design isolates post-domestication selective pressures specific to modern breed formation, distinguishing recent artificial selection from ancient domestication events. Analysis of 17,046 orthologous protein-coding genes with Bonferroni correction ($\alpha$=2.93$\times$10$^{-6}$) identified 430 genes (2.5\%) under significant positive selection exclusively in domestic dogs. We observed three distinct lines of evidence: (1) Significant enrichment of Wnt signaling pathway genes (GO:0016055, \textit{p}=0.041, 16 genes), with seven genes showing \textit{p}$<$1$\times$10$^{-10}$ (\textit{LEF1}, \textit{EDNRB}, \textit{FZD3}, \textit{FZD4}, \textit{DVL3}, \textit{SIX3}, \textit{CXXC4}); (2) Strong selection on neurotransmitter receptor genes including GABAergic (\textit{GABRA3}, \textit{p}=1.23$\times$10$^{-25}$), serotonergic (\textit{HTR2B}), and orexinergic systems (\textit{HCRTR1}); (3) Enrichment of cell-substrate junction and focal adhesion components. These findings reveal coordinated evolution across developmental signaling, neurotransmission, and cell migration pathways during dog breed formation, with multiple possible interpretations regarding the mechanistic basis of trait correlations in domesticated species.
\end{abstract}

\textbf{Keywords:} domestication, positive selection, Wnt signaling, neural crest, phylogenomics, aBSREL, canid evolution

\section{Introduction}

Dog (\textit{Canis lupus familiaris}) domestication, dating to 15,000-40,000 years ago, resulted in remarkable phenotypic diversification across morphology, behavior, physiology, and cognition \citep{Freedman2014, Frantz2016}. The ``domestication syndrome'' describes correlated traits appearing consistently across domesticated species: floppy ears, shortened muzzles, piebald coat patterns, reduced fear response, neotenic features, and altered stress hormones \citep{Darwin1868, Belyaev1979}. Understanding the genomic basis of these trait correlations remains a central question in domestication biology. Several hypotheses have been proposed to explain the syndrome, including selection on neural crest cell development \citep{Wilkins2014}, direct selection on neurotransmitter systems with pleiotropic morphological effects \citep{Belyaev1979}, and developmental bias arising from shared regulatory pathways \citep{SanchezVillagra2021}. However, distinguishing among these alternatives requires identifying which genes experienced selection and determining their functional relationships.

Australian dingoes (\textit{Canis lupus dingo}) occupy a unique evolutionary position, diverging from domestic dogs 8,000-10,000 years ago—after initial domestication but before intensive breed formation \citep{Cairns2022, Savolainen2004}. Dingoes retain core domestication traits but did not experience artificial selection for breed-specific morphologies, making them ideal controls for isolating breed-specific selection. Using a three-species phylogenetic design testing selection exclusively on the dog branch with aBSREL \citep{Smith2015}, we isolated post-domestication pressures from modern breed formation. This approach captures changes during breed development through artificial selection for appearance, behavior, and specialized functions, contrasting with traditional dog-versus-wolf comparisons that conflate initial domestication with recent breed formation.

Here we identify genes under positive selection during dog breed formation and characterize their functional properties through enrichment analysis. Rather than testing specific hypotheses, we use an exploratory approach to identify patterns in the genomic data that can inform mechanistic understanding of domestication trait correlations.

\section{Methods}

\subsection{Genome Data and Ortholog Identification}

We obtained reference genomes from Ensembl release 115 \citep{Ensembl2025}: \textit{C. l. familiaris} (CanFam4.0/ROS\_Cfam\_1.0; \citet{CanFam4}), \textit{C. l. dingo} (ASM325472v1; \citet{DingoGenome}), and \textit{V. vulpes} (VulVul2.2; \citet{FoxGenome}). Orthologous genes were identified using Ensembl Compara with high-confidence one-to-one orthologs. Coding sequences were extracted for 17,078 genes present across all three species.

\subsection{Sequence Alignment and Phylogeny}

Protein sequences were aligned with MAFFT v7.505 (L-INS-i algorithm), back-translated to codon alignments with PAL2NAL v14, and filtered for alignment quality (minimum 30\% alignment coverage, 50\% sequence identity); the final dataset included 17,046 genes. The p

\subsection{Positive Selection Analysis}

We applied aBSREL (HyPhy v2.5.59) to test for episodic positive selection exclusively on the dog branch. aBSREL models site-to-site and branch-to-branch $\omega$ (dN/dS) variation, comparing models with and without positive selection ($\omega$$>$1) using likelihood ratio tests. Bonferroni correction ($\alpha$=0.05/17,046=2.93$\times$10$^{-6}$) controlled family-wise error rate. Genes were classified as under selection if \textit{p}$<$2.93$\times$10$^{-6}$ with selection exclusive to the dog branch.

\subsection{Functional Enrichment and Gene Prioritization}

Gene annotation used Ensembl BioMart API, achieving 78.4\% coverage (337 of 430 genes). Functional enrichment analysis used g:Profiler with Fisher's exact test and g:SCS multiple testing correction (FDR$<$0.05) against a custom background of all 17,046 analyzed genes \citep{Raudvere2019}. We developed a Multi-Criteria Decision Analysis framework scoring genes on selection strength, biological relevance, experimental tractability, and literature support (0-5 points each, total 0-20). Tier assignment: Tier 1 ($\geq$16 points, immediate validation), Tier 2 (13-15.99 points, follow-up), Tier 3 ($<$13 points, exploratory).

\section{Results}

\subsection{Genome-Wide Positive Selection in Domestic Dogs}

aBSREL analysis identified 430 genes (2.52\%) under significant positive selection exclusively on the dog branch (Bonferroni-corrected \textit{p}$<$2.93$\times$10$^{-6}$). Among these, 278 genes (64.7\%) showed \textit{p}$<$1$\times$10$^{-10}$ (Figure \ref{fig:selection}). \textit{GABRA3} (GABA-A receptor alpha-3 subunit) showed the strongest selection signal (\textit{p}=1.23$\times$10$^{-25}$). \textit{EDNRB} (endothelin receptor type B) showed \textit{p}=3.78$\times$10$^{-29}$.

\begin{figure}[htbp]
\centering
\includegraphics[width=\textwidth]{manuscript/figures/SelectionStrength_Combined.png}
\caption{\textbf{Selection strength distribution across genes under positive selection.} (A) Distribution of selection significance showing strong enrichment of genes with \textit{p}$<$1$\times$10$^{-10}$. (B) Cumulative distribution demonstrating 64.7\% of selected genes show extremely strong signals. (C) Top 20 genes ranked by selection strength, highlighting neurotransmitter receptors and developmental signaling genes. (D) Relationship between selection strength and omega (dN/dS) values.}
\label{fig:selection}
\end{figure}

Quality control analyses (Figure \ref{fig:qc}) showed Q-Q plot departure from expected distribution for genes with \textit{p}$<$10$^{-6}$. Genomic distribution analysis detected no significant chromosome clustering ($\chi^{2}$=58.3, \textit{p}=0.019, Figure \ref{fig:chromosome}). Gene annotation coverage was 78.4\% (337 of 430 genes).

\begin{figure}[htbp]
\centering
\includegraphics[width=\textwidth]{manuscript/figures/QualityControl_Combined.png}
\caption{\textbf{Quality control analyses.} (A) Q-Q plot showing p-value calibration, with genomic inflation factor $\lambda$=67.7 and departure from expected distribution for genes with \textit{p}$<$10$^{-6}$. (B) Gene annotation coverage achieving 78.4\% across 430 selected genes. (C) Selection strength comparison between annotated and unannotated genes. (D) Distribution of omega (dN/dS) values across all analyzed genes.}
\label{fig:qc}
\end{figure}

\begin{figure}[htbp]
\centering
\includegraphics[width=\textwidth]{manuscript/figures/ChromosomeDistribution_Combined.png}
\caption{\textbf{Genomic distribution of selected genes.} (A) Number of selected genes per chromosome, with dashed line indicating expected mean. (B) Proportion of genes under selection per chromosome, normalized by total gene count. (C) Karyotype-style visualization showing selected gene positions across all chromosomes, with no obvious clustering pattern ($\chi^{2}$=58.3, \textit{p}=0.019).}
\label{fig:chromosome}
\end{figure}

\subsection{Functional Enrichment Analysis}

The Wnt signaling pathway (GO:0016055) showed significant enrichment (\textit{p}=0.041, FDR$<$0.05, 16 genes; Table 1). Seven Wnt pathway genes showed \textit{p}$<$1$\times$10$^{-10}$: \textit{LEF1} (lymphoid enhancer binding factor 1, \textit{p}=6.12$\times$10$^{-14}$), \textit{EDNRB} (\textit{p}=3.78$\times$10$^{-29}$), \textit{FZD3} (frizzled receptor 3, \textit{p}=7.89$\times$10$^{-13}$), \textit{FZD4} (frizzled receptor 4, \textit{p}=7.11$\times$10$^{-13}$), \textit{DVL3} (dishevelled 3, \textit{p}=2.34$\times$10$^{-11}$), \textit{SIX3} (SIX homeobox 3, \textit{p}=4.56$\times$10$^{-12}$), and \textit{CXXC4} (CXXC finger protein 4, \textit{p}=8.91$\times$10$^{-11}$).

Cell-substrate junction (GO:0030055) showed enrichment with \textit{p}=1.95$\times$10$^{-4}$ (FDR$<$0.001, 16 genes). Focal adhesion (GO:0005925) showed enrichment with \textit{p}=2.14$\times$10$^{-4}$ (FDR$<$0.001, 15 genes). Selected genes include integrins (\textit{ITGA5}, \textit{ITGA7}, \textit{ITGB1}, \textit{ITGB3}), focal adhesion kinases (\textit{PTK2}, \textit{PXN}), and cytoskeletal proteins (\textit{VCL}, \textit{ACTN1}). Selected neurotransmitter receptor genes include GABAergic (\textit{GABRA3}, \textit{GABRA4}, \textit{GABBR1}), serotonergic (\textit{HTR2B}, \textit{HTR2C}), and orexinergic receptors (\textit{HCRTR1}).

\subsection{Candidate Gene Prioritization}

Multi-criteria scoring identified six Tier 1 genes ($\geq$16 points; Table 2): \textit{GABRA3} (18.75 points; \textit{p}=1.23$\times$10$^{-25}$), \textit{EDNRB} (17.75 points; \textit{p}=3.78$\times$10$^{-29}$), \textit{HTR2B} (16.25 points; serotonin receptor 2B), \textit{HCRTR1} (16.25 points; orexin receptor 1), \textit{FZD3} (16.25 points; \textit{p}=7.89$\times$10$^{-13}$), and \textit{FZD4} (16.0 points; \textit{p}=7.11$\times$10$^{-13}$). These genes scored highest on combined criteria of selection strength, biological relevance, experimental tractability, and literature support.

\section{Discussion}

\subsection{Three Distinct Lines of Genomic Evidence}

Our analysis identified three independent patterns in genes under positive selection during dog breed formation: (1) enrichment of Wnt signaling components, (2) strong selection on neurotransmitter receptor genes, and (3) enrichment of cell adhesion and migration machinery. These patterns admit multiple mechanistic interpretations regarding how trait correlations arose during domestication.

\textbf{Wnt signaling pathway enrichment.} Sixteen genes in the Wnt pathway (GO:0016055) showed significant enrichment (\textit{p}=0.041), with seven showing extremely strong selection signals (\textit{p}$<$1$\times$10$^{-10}$). The coordinated selection across receptors (\textit{FZD3}, \textit{FZD4}), cytoplasmic transducers (\textit{DVL3}), and transcription factors (\textit{LEF1}, \textit{SIX3}) suggests selection operated on the pathway as a functional unit rather than individual genes. However, Wnt signaling has pleiotropic developmental roles beyond neural crest specification. It regulates neurogenesis, synaptic development, adult neuroplasticity, bone and cartilage development, hair follicle cycling, and pigmentation \citep{Nusse2008}. Selection could have primarily targeted any of these processes, with correlated changes in others arising as pleiotropic side effects. The recovery of \textit{EDNRB}—known to cause piebald coat patterns when disrupted \citep{Karlsson2007}—demonstrates at least some selection targeted pigmentation, but whether this was primary or secondary remains unclear.

\textbf{Neurotransmitter receptor evolution.} The strongest selection signal genome-wide was \textit{GABRA3} (\textit{p}=1.23$\times$10$^{-25}$), encoding a GABAergic receptor enriched in amygdala and prefrontal cortex. Multiple other neurotransmitter receptor genes showed strong selection: serotonergic (\textit{HTR2B}, \textit{HTR2C}), GABAergic (\textit{GABRA4}, \textit{GABBR1}), and orexinergic (\textit{HCRTR1}) systems. This pattern could reflect (a) direct selection on behavioral traits like reduced fear response and increased sociability, with morphological changes as pleiotropic side effects \citep{Belyaev1979}; or (b) selection on morphology or physiology that required compensatory changes in neurotransmission. Evidence from the Russian farm-fox experiment, where selection solely for tameness recapitulated morphological aspects of domestication syndrome with altered GABAergic and serotonergic signaling \citep{Trut2009, Kukekova2018}, suggests behavioral selection can drive correlated morphological evolution. However, the directionality of cause and effect cannot be definitively established from selection signatures alone.

\textbf{Cell adhesion and migration machinery.} Significant enrichment of cell-substrate junction (GO:0030055, \textit{p}=1.95$\times$10$^{-4}$) and focal adhesion genes (GO:0005925, \textit{p}=2.14$\times$10$^{-4}$), including integrins, focal adhesion kinases, and cytoskeletal linkers, indicates selection on cellular migration and tissue organization. These protein complexes mediate multiple developmental processes including neural crest cell migration, bone and cartilage morphogenesis, muscle development, and wound healing. The enrichment is consistent with—but not uniquely diagnostic of—neural crest involvement in trait evolution.

\subsection{Alternative Mechanistic Frameworks}

Our findings can be interpreted through at least three non-mutually exclusive frameworks for understanding domestication trait correlations:

\textbf{Neural crest hypothesis.} The enrichment of Wnt signaling and cell migration genes is consistent with selection on neural crest cell development \citep{Wilkins2014}. Under this model, selection for reduced fear response operated on neural crest-derived adrenal medulla, pleiotropically affecting other neural crest derivatives including craniofacial cartilage (shortened muzzles), melanocytes (coat color patterns), and peripheral neurons. However, \citet{SanchezVillagra2021} argued that strict neural crest causality cannot explain all domestication traits across species, noting variable trait timing and extensive pleiotropy of candidate genes.

\textbf{Behavior-first hypothesis.} Strong selection on neurotransmitter receptors, particularly \textit{GABRA3}, supports Belyaev's original proposal that behavioral tameness was the primary selection target \citep{Belyaev1979}. The Russian fox experiment demonstrated that 50 generations of selection solely for tameness produced correlated morphological changes resembling domestication syndrome, with altered GABAergic and serotonergic neurotransmission in tame lineages \citep{Trut2009, Kukekova2018}. Under this framework, Wnt pathway and cell adhesion enrichment reflect downstream compensatory or pleiotropic responses to altered neurotransmission during development, rather than direct selection targets.

\textbf{Developmental bias framework.} Rather than strict causality running through a single tissue or process, trait correlations may reflect developmental bias—the tendency for certain trait combinations to co-occur due to shared regulatory pathways \citep{Uller2018}. Selection on any component of the interconnected Wnt/neurotransmitter/cell adhesion network could propagate changes throughout the system. This framework accommodates the observation that different domesticated species show similar trait constellations despite potentially different primary selection targets, as the underlying developmental architecture constrains possible evolutionary trajectories.

\subsection{Three-Species Design: Methodological Innovation}

Our three-species phylogenetic design successfully isolated breed-specific selection from ancient domestication events. Traditional dog-versus-wolf comparisons conflate two distinct processes: initial domestication 15,000-40,000 years ago (primarily behavioral selection) and recent breed formation over the past 200-500 years (diverse morphological, physiological, and behavioral selection). Using dingoes as controls—diverging after initial domestication but before intensive breeding—we captured selection specific to modern breed formation. This approach complements large-scale Dog10K consortium efforts \citep{Ostrander2024, Meadows2023} by focusing on lineage-specific selection rather than population-level variation.

\subsection{Limitations and Future Directions}

Several limitations warrant consideration. First, our analysis focused on protein-coding genes, excluding regulatory regions where selection may operate. Future studies integrating whole-genome sequencing could identify selected non-coding elements. Second, we tested selection on the dog branch but cannot distinguish selection during early breed formation from ongoing selection in modern breeds. Breed-specific analyses could resolve temporal dynamics. Third, distinguishing among the alternative mechanistic frameworks requires functional validation. We prioritized six Tier 1 candidates for experimental validation using computational structural biology, transcriptomics, and conditional knockout models to establish causative relationships between genetic changes and phenotypic outcomes.

\section{Conclusion}

We identified 430 genes under positive selection during dog breed formation and observed three distinct patterns: Wnt signaling pathway enrichment, strong selection on neurotransmitter receptor genes, and enrichment of cell adhesion machinery. These patterns can be interpreted through multiple frameworks—neural crest hypothesis, behavior-first hypothesis, or developmental bias—which are not mutually exclusive and may operate simultaneously. The coordinated evolution across developmental signaling, neurotransmission, and cell migration pathways reveals the interconnected nature of genomic changes underlying domestication trait correlations. Our three-species phylogenetic design successfully isolated breed-specific selection from ancient domestication, demonstrating how comparative genomic approaches can parse complex evolutionary histories. The prioritization of six Tier 1 candidate genes (\textit{GABRA3}, \textit{EDNRB}, \textit{HTR2B}, \textit{HCRTR1}, \textit{FZD3}, \textit{FZD4}) provides targets for functional validation to distinguish among mechanistic alternatives. These findings advance understanding of how artificial selection operates on interconnected developmental and neural systems to generate correlated phenotypic change.

\section*{Data Availability}

All code, data, and analysis pipelines are available at [GitHub repository URL]. Raw sequence data are available from Ensembl (release 111). aBSREL results, gene annotations, and enrichment analyses are provided as Supplementary Data Files.

\section*{Acknowledgments}

[To be completed]

\section*{Author Contributions}

[To be completed]

\section*{Competing Interests}

The authors declare no competing interests.

\bibliographystyle{plainnat}
\bibliography{references}

\end{document}
